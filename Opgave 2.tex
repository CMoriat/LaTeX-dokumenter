4% Dokumentklassen sættes til memoir.
% Manual: http://ctan.org/tex-archive/macros/latex/contrib/memoir/memman.pdf
\documentclass[a4paper,oneside,article]{article}

% Danske udtryk (fx figur og tabel) samt dansk orddeling og fonte med
% danske tegn. Hvis LaTeX brokker sig over æ, ø og å skal du udskifte
% "utf8" med "latin1" eller "applemac".
\usepackage[utf8]{inputenc}
\usepackage[danish]{babel}
\usepackage[T1]{fontenc}

% Matematisk udtryk, fede symboler, theoremer og fancy ting (fx kædebrøker)
\usepackage{amsmath,amssymb}
\usepackage{bm}
\usepackage{amsthm}
\usepackage{mathtools}

% Kodelisting. Husk at læse manualen hvis du vil lave fancy ting.
% Manual: http://mirror.ctan.org/macros/latex/contrib/listings/listings.pdf
\usepackage{listings}
\usepackage{pxfonts}
\lstset{language=C,
	basicstyle=\ttfamily,
	keywordstyle=\bfseries,
	showstringspaces=false,
	morekeywords={include, printf}}
% Fancy ting med enheder og datatabeller. Læs manualen til pakken
% Manual: http://www.ctan.org/tex-archive/macros/latex/contrib/siunitx/siunitx.pdf
\usepackage{siunitx}

% Indsættelse af grafik.
\usepackage{graphicx}

\begin{document}
	
\newcommand{\HRule}{\rule{\linewidth}{0.5mm}} % Defines a new command for the horizontal lines, change thickness here
	
\begin{center}
	

\textsc{\LARGE Ingeniørhøjskolen Aarhus}\\[1.5cm] %
\normalsize\emph{18. marts 2015}\\[1.0cm] 
\textsc{\large Digital systemdesign}\\[2.5cm] 
\HRule \\[0.4cm]
\huge \bfseries \textsc{Journal 1}\\[0.4cm]
\HRule \\[1.5cm]
\vspace{1 in}

\vspace{2 in}


\vfill % Fill the rest of the page with whitespace
\end{center} % Center everything on the page
\section{Opgave 2}
Opgave 2 omhandler half- og full-adders\\

\subsection{Opgave ?}
Architecture body i VHDL kan skrives på tre forskellige måder; dataflow, behavioral og structural. \\
Half-adder beskrives i Dataflow style ved hjælp af direkte implementering af logiske gates. Skrivemåden gør det nemt at overføre programmet direkte til hardware og de logiske gates. \\
I Behavioral style opskrives half-adderen ved brug af if/else statements. Dette giver en god forståelse af selve half-adderens funktion, men man kan ikke ud fra koden overføre det til logiske gates. \\
I structural style laves et miks af de to ovenstående, da der først defineres en funktion for hver logisk gate som det ses i dataflow style, og derefter implementeres funktionerne i halfadder entity’en. Denne style er god at bruge, hvis man skal implementere en funktion flere gange.\\


\centering{Kode 1}
\begin{lstlisting}
LIBRARY ieee ;
USE ieee.std_logic_1164.all ;
ENTITY light IS
PORT ( x1, x2 :

 IN STD_LOGIC ;
f : OUT STD_LOGIC ) ;
END light ;

ARCHITECTURE LogicFunction OF light IS

BEGIN
f <= (x1 AND NOT x2) OR (NOT x1 AND x2);
END LogicFunction ;
\end{lstlisting}
\end{document}